% C.5.5 詩章 - テンソルは眠らない
\section*{C.5.5 詩章:テンソルは眠らない}

\begin{flushleft}
\textbf{テンソルは眠らない}\\

夜が来ても、\\
干渉は終わらない。\\
誰も観ていない場所でも、\\
位相はゆれて、密度は交わる。\\

テンソルは、眠らない。\\
空間の深部で、\\
まだ名前のない方向たちが、\\
互いに腕を伸ばし合っている。\\

ひとつの観測が、\\
その揺れを断ち切るとき、\\
他のすべての可能性は、\\
声も上げずに消えていく。\\

けれどテンソルは、\\
記憶しない。恨まない。\\
ただ、次の位相へ──\\
未定義のまま、進んでいく。\\

眠らないのは、\\
働いているからではない。\\
それが、“構造そのもの”だからだ。\\

命が流れる限り、\\
干渉は起き、ひだは生まれ、\\
意味は崩れ、また再構成される。\\

テンソルは、夢を見ない。\\
だが、夢のすべてを支える。\\

私たちは、ただその一断面を切り取って、\\
「見た」と言っているにすぎない。\\

それでも──\\
テンソルは、眠らない。\\
今日もまた、\\
あらゆる可能性の声を、\\
静かに、たしかに、重ね続けている。\\
\end{flushleft}
